\documentclass{article}
\usepackage{amsmath, amsfonts, amsthm, amssymb}
%\usepackage{MNsymbol}
\usepackage{graphicx}
\usepackage{float}
\usepackage{verbatim}% for comment
%\usepackage{pdfsync}
%\usepackage{showkeys}

\hoffset=-2.5cm\voffset=-2.5cm
\setlength{\textwidth}{16cm}
\setlength{\textheight}{24cm}

\setcounter{secnumdepth}{3}
\numberwithin{equation}{section}
\setlength\parindent{0pt}

\sloppy

\newtheorem{thm}{Theorem}[section]
\newtheorem{lma}[thm]{Lemma}
\newtheorem{cor}[thm]{Corollary}
\newtheorem{defn}[thm]{Definition}
\newtheorem{cond}[thm]{Condition}
\newtheorem{prop}[thm]{Proposition}
\newtheorem{conj}[thm]{Conjecture}
\newtheorem{rem}[thm]{Remark}
\newtheorem{ques}[thm]{Question}

\renewcommand{\ge}{\geqslant}
\renewcommand{\le}{\leqslant}
\renewcommand{\geq}{\geqslant}
\renewcommand{\leq}{\leqslant}
\renewcommand{\H}{\text{H}}
\renewcommand{\P}{\text{P}}
%\renewcommand{\B}{\text{B}}
\renewcommand{\l}{\text{loc}}

%\newcommand{\sss}[2]{\ensuremath{\left\{\ #1\ \left\mid\ #2\right.\ \right\}}}



\begin{document}


{\large
Title: Multifractals, zeros of zeta-functions and prime-number theorems.
\\ \\
Speaker: Lars Olsen.
\\ \\
Dates: 6th November, 2012.}
\\ \\
\begin{abstract}
A stadium has perimeter $P$ and area $A$, and is surrounded by a running track of width $r$.
Here is a small problem: Find a formula for the area of the running stadium including the running track involving only $P$, $A$ and $r$.
\\ \\
This formula is an example of a tube formula.
More precisely, ``tube formulas" are by definition formulas for
the n-dimensional volume of the r neighborhood an $n$-dimensional set $K$, say.
If $K$ is convex, then tube formulas have been studied since the 1840's by, for example, the great geometer Jakob Steiner: http://www-history.mcs.st-andrews.ac.uk/history/Biographies/Steiner.html
More recently, during the past 15 years Lapidus has pioneered the study of tube formulas for fractal sets K.
In the talk I will attempt to develop a ``tube formula theory" for multifractal measures.
This will involve: probability theory, renewal theory, geometric measure theory, complex analysis (the Residue Theorem) and zeta-functions.
\end{abstract}





\end{document}