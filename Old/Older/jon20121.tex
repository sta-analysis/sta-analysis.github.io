\documentclass{article}
\usepackage{amsmath, amsfonts, amsthm, amssymb}
%\usepackage{MNsymbol}
\usepackage{graphicx}
\usepackage{float}
\usepackage{verbatim}% for comment
%\usepackage{pdfsync}
%\usepackage{showkeys}

\hoffset=-2.5cm\voffset=-2.5cm
\setlength{\textwidth}{16cm}
\setlength{\textheight}{24cm}

\setcounter{secnumdepth}{3}
\numberwithin{equation}{section}
\setlength\parindent{0pt}

\sloppy

\newtheorem{thm}{Theorem}[section]
\newtheorem{lma}[thm]{Lemma}
\newtheorem{cor}[thm]{Corollary}
\newtheorem{defn}[thm]{Definition}
\newtheorem{cond}[thm]{Condition}
\newtheorem{prop}[thm]{Proposition}
\newtheorem{conj}[thm]{Conjecture}
\newtheorem{rem}[thm]{Remark}
\newtheorem{ques}[thm]{Question}

\renewcommand{\ge}{\geqslant}
\renewcommand{\le}{\leqslant}
\renewcommand{\geq}{\geqslant}
\renewcommand{\leq}{\leqslant}
\renewcommand{\H}{\text{H}}
\renewcommand{\P}{\text{P}}
%\renewcommand{\B}{\text{B}}
\renewcommand{\l}{\text{loc}}

%\newcommand{\sss}[2]{\ensuremath{\left\{\ #1\ \left\mid\ #2\right.\ \right\}}}




\begin{document}


{\large
Title: Assouad type dimensions and homogeneity of fractal sets.
\\ \\
Speaker: Jonathan Fraser.
\\ \\
Dates: 9th October, 2012.}
\\ \\
\begin{abstract}
I will discuss two dual notions of dimension, namely, the \emph{Assouad dimension} and the \emph{lower dimension}.  Unlike the more popular Hausdorff, packing and box dimensions, these dimensions depend only on the extreme local scaling rather than some sort of average local scaling.  As such they give coarse, but easily interpreted geometric information about a fractal set.  I will begin by discussing some of their (sometimes surprising) basic properties and then go on to calculate them explicitly for several classes of sets with varying degrees of homogeneity.
\end{abstract}





\end{document}