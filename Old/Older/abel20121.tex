\documentclass{article}
\usepackage{amsmath, amsfonts, amsthm, amssymb}
%\usepackage{MNsymbol}
\usepackage{graphicx}
\usepackage{float}
\usepackage{verbatim}% for comment
%\usepackage{pdfsync}
%\usepackage{showkeys}

\hoffset=-2.5cm\voffset=-2.5cm
\setlength{\textwidth}{16cm}
\setlength{\textheight}{24cm}

\setcounter{secnumdepth}{3}
\numberwithin{equation}{section}
\setlength\parindent{0pt}

\sloppy

\newtheorem{thm}{Theorem}[section]
\newtheorem{lma}[thm]{Lemma}
\newtheorem{cor}[thm]{Corollary}
\newtheorem{defn}[thm]{Definition}
\newtheorem{cond}[thm]{Condition}
\newtheorem{prop}[thm]{Proposition}
\newtheorem{conj}[thm]{Conjecture}
\newtheorem{rem}[thm]{Remark}
\newtheorem{ques}[thm]{Question}

\renewcommand{\ge}{\geqslant}
\renewcommand{\le}{\leqslant}
\renewcommand{\geq}{\geqslant}
\renewcommand{\leq}{\leqslant}
\renewcommand{\H}{\text{H}}
\renewcommand{\P}{\text{P}}
%\renewcommand{\B}{\text{B}}
\renewcommand{\l}{\text{loc}}

%\newcommand{\sss}[2]{\ensuremath{\left\{\ #1\ \left\mid\ #2\right.\ \right\}}}




\begin{document}


{\large
Title: Linear images of self similar sets.
\\ \\
Speaker: \'Abel Farkas
\\ \\
Date: 13th November, 2012.}
\\ \\
\begin{abstract}
I will discuss the structure of linear images of self-similar sets. In particular, I will study the behaviour of the dimension and measure and go on to develop a method that helps to handle problems in situations where the open set condition is not satisfied.

\end{abstract}




\end{document}